\documentclass{article}

\def\Sumj{{{\sum_{j=1}^{J}}}}

\pdfpagewidth 8.5in
\pdfpageheight 11in
\setlength\topmargin{-0.3in}
\setlength\textheight{9in}
\setlength\textwidth{6.5in}
\setlength\oddsidemargin{0in}
\setlength\evensidemargin{0in}

\begin{document}
\begin{center}
{\huge PERMUTATION TEST\\}
\end{center}
\vspace{1cm}
It is useful to assess the strength of evidence of an association between the
response variable and the prognostic variable (categorized into a binary 
variable using a cutoff value obtained from the results of the binary 
partitioning algorithm).

Partitioning methods in which splits are determined by the use of p-values are afflicted by the problem of multiple testing which arises due to the repeated `looks' at the data that are involved in obtaining the optimal split \cite{MG00}.  Approaching binary partitioning from a hypothesis testing framework necessitates that the multiple testing issue be addressed; typically some penalty is applied, similar to a Bonferroni adjustment in the case of multiple hypotheses being tested.

Since we take the approach of obtaining an optimal split of the data
based on a goodness-of-split measure, we are not in the same hypothesis
testing scenario as that described above.  Therefore we must approach this
issue from a different angle.

A distribution-free permutation test \cite{CH74} is used to assess the strength of
evidence of an association between the prognostic variable which has been
dichotomized into `high' and `low' levels via the binary partitioning method
proposed in this paper and the response variable.  Within the same framework
as presented by Kim et al. \cite{KFFM00} we implement the permutation test in 
the following several steps:
\begin{description}
\item[1.]  Calculate $\Delta D_{observed} (s^{*},{\cal N})$, the optimal 
goodness-of-fit measure obtained from the observed data.
\item[2.] Permute the observed data to obtain $J$ permutation data sets 
${\cal N}_{j}$, where $j=1, \cdots, J$. This is simply achieved by randomly
permuting the covariate values while holding the response variable fixed.
\item[3.] For each of these data sets, compute the optimal goodness-of-fit 
measure $\Delta D_{j} (s^{*},{\cal N}_{j})$, where $j=1, \cdots, J$.
\item[4.] The p-value is obtained from the permutational distribution of the
goodness-of-fit statistics.  In fact, the p-value can be simply calculated as 
\begin{eqnarray*}
p = \frac{ \Sumj \left[ I \left( \Delta D_{j} (s^{*}, {\cal N}_{j}) \geq 
	\Delta D_{observed} (s^{*},{\cal N}) \right) \right] }{J}.
\end{eqnarray*}
\end{description}

We sample from the permutation distribution because it is typically computationally 
prohibitive to obtain $n!$ permutations of the observed data and to compute an 
optimal goodness-of-fit measure for each of those data sets. Therefore, as
an approximation, one can chose $J$ sufficiently large in order to achieve as 
many significant digits as desired for the p-value.

\begin{thebibliography}{99}
\bibitem{MG00}
Mazumdar M, Glassman JR.
Tutorial in Biostatistics --  Categorizing a prognostic variable: review of methods, code for easy implementation and applications to decision-making about cancer treatments.
\textit{Statistics in Medicine} 2000;
\textbf{19}:113--132.

\bibitem{CH74}
Cox DR, Hinkley DV.
\textit{Theoretical Statistics}
Chapman and Hall: London, 1974.

\bibitem{KFFM00}
Kim H-J, Fay MP, Feuer EJ, Midthune DN.
Permutation tests for joinpoint regression with applications to cancer rates.
\textit{Statistics in Medicine} 2000;
\textbf{19}:335-351.
\end{thebibliography}
\end{document}